\documentclass{beamer}
\usetheme{Copenhagen}

\title{LWE and cryptography}

\begin{document}

\frame{
	\titlepage
}

\frame{
	\frametitle{Motivation}
	\begin{itemize}[<+->]
		\item Originally as an attempt to solve lattice problems
		\item Worst-case lattice problems reduce to LWE
		\item Need new encryption schemes that are at least as hard to
			break as problems difficult for quantum computers
		\item LWE fits the bill
	\end{itemize}
}

\frame{
	\frametitle{Intuition of LWE}
	$$14s_1+15s_2+5s_3+2s_4\approx8\pmod{17}$$
	$$13s_1+14s_2+14s_3+6s_4\approx16\pmod{17}$$
	$$6s_1+10s_2+13s_3+s_4\approx3\pmod{17}$$
	$$10s_1+4s_2+12s_3+16s_4\approx12\pmod{17}$$
	$$9s_1+5s_2+9s_3+6s_4\approx9\pmod{17}$$
	$$3s_1+6s_2+4s_3+5s_4\approx16\pmod{17}$$
	$$\vdots$$
	$$6s_1+7s_2+16s_3+2s_4\approx3\pmod{17}$$
}

\frame{
	\frametitle{Definition (LWE Distribution)}
	Let $A_{\vec{s}, \chi}$ be a distribution on
	$\mathbb{Z}^n_q\times\mathbb{Z}_q$ as follows:
	\pause
	\begin{itemize}[<+->]
		\item Pick $\vec{a}\in\mathbb{Z}^n_q$ uniformly randomly
		\item Pick $e$ according to $\chi$
		\item Output $(\vec{a}, \langle\vec{a}, \vec{s}\rangle + e)$
	\end{itemize}
}

\frame{
	\frametitle{Definition (LWE)}
	Given samples from $A_{\vec{s}, \chi}$,
	\begin{itemize}[<+->]
		\item Search version: Find $\vec{s}$.
		\item Decision version: Distinguish between $A_{\vec{s}, \chi}$
			and the uniform distribution.
	\end{itemize}
}


\frame{
	\frametitle{Search to decision reduction}


}

\frame{
	\frametitle{GSW's LWE formulation}
	\begin{enumerate}[<+->]
		\item Take the rows of $A$ as $b_i||\vec{a}_i$ instead
		\item Redefine $\vec{s}$ as $(1,-\vec{s})$
		\item Either $A$ is uniform, or $\exists\vec{s}$ with
			first coefficient 1 s.t. $A\vec{s}=\vec{e}$,
			where $\vec{e}$ comes from $\chi$.
	\end{enumerate}
}

\frame{
	\frametitle{Asymmetric cryptography}
	Two parties who may have never communicated before may securely
	exchange information, by using the tools below:
	\pause
	\begin{itemize}[<+->]
		\item $pk$: ``public key", used for encryption
		\item $\textnormal{Enc}_{pk}:\Sigma^*\times\mathcal{R}\rightarrow U$
		\item $sk$: ``secret key", used for decryption
		\item $\textnormal{Dec}_{sk}:U\rightarrow\Sigma^*$
		\item $\textnormal{Dec}_{sk}\circ\textnormal{Enc}_{pk}(x, r)=x$
			with overwhelming probability over $r$
	\end{itemize}
}

\frame{
	Typical use of such a scheme:
	\frametitle{Asymmetric cryptography cont.}
	\begin{enumerate}[<+->]
		\item Alice generates $(pk, sk)$
		\item Alice sends $pk$ to Bob
		\item Bob encrypts his message using $pk$
		\item Bob sends the ciphertext to Alice
		\item Alice decrypts it using $sk$
	\end{enumerate}
}

\frame{
	\frametitle{Safety}
	What does it mean for an encryption scheme to be ``safe"?
	\pause
	\begin{itemize}[<+->]
		\item Chosen plaintext attack (CPA): The "intuitive" definition.
			An (efficient) adversary who's able to encrypt anything shouldn't
			be able to decrypt anything.
		\item Adaptive chosen-ciphertext attack (CCA2):
			A stronger definition. An (efficient) adversary who's also able to
			decrypt anything but the target, still cannot decrypt the target.
	\end{itemize}
}

\frame{
	\frametitle{Safety (Malleability)}
	A cryptographic scheme is malleable if
	$\exists f:\Sigma^*\rightarrow\Sigma^*$ efficiently invertible,
	an entity given $pk$ and $\textnormal{Enc}_{pk}(x)$
	can evaluate $\textnormal{Enc}_{pk}(f(x))$.
	\pause
	\begin{itemize}[<+->]
		\item CCA2 implies non-malleability
		\item Has many flavors (malleable under CPA vs CCA)
		\item Is this always a bad property to have?
	\end{itemize}
}

\frame{
	\frametitle{Homomorphic encryption}
	Let someone else do the computation for you.
	Useful when that ``someone else" is quantum!
	\pause
	\begin{enumerate}[<+->]
		\item Alice generates $(pk, sk, evk)$
		\item Alice encrypts her message using $pk$
		\item Alice sends $evk$ and the ciphertext to Bob
		\item Bob runs computations on the ciphertext
		\item Bob sends the encrypted result back to Alice
		\item Alice decrypts it using $sk$.
	\end{enumerate}
}

\frame{
	\frametitle{Idea of GSW's homomorphic encryption}
	Intuitive idea as follows:
	\begin{itemize}[<+->]
		\item private key $\vec{v}$ is a vector
		\item ciphertexts are matrices with $\vec{v}$ approximately as an
			eigenvector
		\item plaintexts are corresponding approximate eigenvalues
		\item $(C_1+C_2)\vec{v}\approx(\lambda_1+\lambda_2)\vec{v}$
		\item $(C_1C_2)\vec{v}\approx(\lambda_1\lambda_2)\vec{v}$
		\item When plaintexts are booleans, $I_N-C_1C_2$ encodes NAND.
	\end{itemize}
}

\frame{
	\frametitle{GSW's tools}
	Define the following functions on $\mathbb{Z}_q^*$.
	Easier to understand with examples.
	\pause
	\begin{itemize}[<+->]
		\item $\textnormal{BitDecomp}(1001_2, 0010_2, 1100_2)
			=(1,0,0,1,0,0,1,0,1,1,0,0)$
		\item $\textnormal{BitDecomp}^{-1}(1,0,0,1,0,0,1,0,1,1,0,0)
			=(1001_2, 0010_2, 1100_2)$
		\item $\textnormal{BitDecomp}^{-1}(0,0,10_2,0,0,1,10_2,1,11_2,0,1,1)
			=(0100_2, 1001_2, 1011_2)$
		\item $\textnormal{Flatten}=\textnormal{BitDecomp}\circ
			\textnormal{BitDecomp}^{-1}$
		\item $\textnormal{Powersof2}(11_2, 0_2, 1_2)=
			(11_2, 110_2, 1100_2, 11000_2,
			0_2, 0_2, 0_2, 0_2, 1_2, 10_2, 100_2, 1000_2)$
	\end{itemize}
}

\frame{
	\frametitle{GSW's tools (Flatten)}
		\begin{align*}
			\textnormal{Flatten}(&110_2, 101_2, 1_2, 11_2,\\
				&110_2, 101_2, 1_2, 11_2,\\
				&110_2, 101_2, 1_2, 11_2)
		\end{align*}
		\pause
		\begin{align*}
		=\textnormal{BitDecomp}\circ
			\textnormal{BitDecomp}^{-1}(&110_2, 101_2, 1_2, 11_2,\\
			&110_2, 101_2, 1_2, 11_2,\\
			&110_2, 101_2, 1_2, 11_2)
		\end{align*}
		\pause
		\begin{align*}
			=\textnormal{BitDecomp}(
			&110000_2+10100_2+10_2+11_2,\\
			&110000_2+10100_2+10_2+11_2,\\
			&110000_2+10100_2+10_2+11_2)
		\end{align*}
}

\frame{
	\frametitle{GSW's tools (Flatten)}
	\begin{align*}
		=\textnormal{BitDecomp}(&1001001_2,\\
		&1001001_2,\\
		&1001001_2)
	\end{align*}
	\pause
	\begin{align*}
	=\textnormal{BitDecomp}(1001_2, 1001_2, 1001_2)
	\end{align*}
	\pause
	\begin{align*}
	=(1,0,0,1,1,0,0,1,1,0,0,1)
	\end{align*}
}

\frame{
	\frametitle{GSW's tools cont.}
	Some basic properties
	\begin{itemize}[<+->]
		\item $\langle\textnormal{BitDecomp}(\vec{a}),
			\textnormal{Powersof2}(\vec{b})\rangle
			=\langle\vec{a},\vec{b}\rangle$
		\item $\langle\vec{a},\textnormal{Powersof2}(\vec{b})\rangle
			=\langle\textnormal{BitDecomp}^{-1}(\vec{a}),\vec{b}\rangle$
		\item $=\langle\textnormal{Flatten}(\vec{a}),
			\textnormal{Powersof2}(\vec{b})\rangle$
	\end{itemize}
}

\frame{
	\frametitle{GSW's Construction - Setup}
	Choose the following parameters:
	\begin{itemize}[<+->]
		\item $q$ of $\kappa(\lambda, L)$ bits
		\item $n(\lambda, L)$ lattice dimension parameter
		\item Error distribution $\chi(\lambda, L)$
		\item $m(\lambda, L)\in O(n\log q)$
		\item $l=1 + \log q$
		\item $N=(n+1)\times l$
	\end{itemize}
}

\frame{
	\frametitle{GSW's Construction - Secret Keygen}
	\begin{enumerate}[<+->]
		\item Sample $\vec{t}\leftarrow\mathbb{Z}^n_q$ uniformly.
			This represents the solution of the LWE system of equations.
		\item Output $sk=\vec{s}=1||-\vec{t}\in\mathbb{Z}^{n+1}_q$
		\item Let $\vec{v}=\textnormal{Powersof2}(\vec{s})$
	\end{enumerate}
}

\frame{
	\frametitle{GSW's Construction - Public Keygen}
	\begin{enumerate}[<+->]
		\item Generate $B\leftarrow\mathbb{Z}^{m\times n}_q$ uniformly
		\item Sample $\vec{e}\leftarrow\chi^m$
		\item Set $\vec{b}=B\vec{t}+\vec{e}$
		\item Set $pk$ as $\vec{b}$ on the first column,
			followed by the columns of $B$.
		\item Observe that $A\vec{s}=\vec{e}$
	\end{enumerate}
}

\frame{
	\frametitle{GSW's Construction - Encryption}
	Input: $\mu$
	\begin{enumerate}[<+->]
		\item Sample $R\in\{0, 1\}^{N\times m}$
		\item Output $\textnormal{Flatten}(
			\mu\cdot I+\textnormal{BitDecomp}(R\cdot A)$
	\end{enumerate}

}

\frame{
	\frametitle{References}
	\begin{itemize}
		\item O. Regev. The Learning with Errors Problem.
		\item C. Gentry, A. Sahai, B. Waters.
			Homomorphic Encryption from Learning with Errors:
			Conceptually-Simpler, Asymptotically-Faster, Attribute-Based.
	\end{itemize}
}

\end{document}
